\documentclass[12pt]{article}
\usepackage{multirow}
\usepackage{caption}
\usepackage{bm}
\usepackage{amsmath}
\usepackage{amsfonts}
\usepackage{amssymb}
\usepackage{graphicx}
\usepackage{colortbl}
\usepackage{xr}
\usepackage{hyperref}
\usepackage[all]{hypcap} 
\usepackage{longtable}
\usepackage{xfrac}
\usepackage{tabularx}
\usepackage{float}
\usepackage{siunitx}
\usepackage{booktabs}
\usepackage[toc,page]{appendix}
\usepackage{url}
\usepackage[usenames,dvipsnames]{xcolor}
%\usepackage{refcheck}


\hypersetup{
      colorlinks=true,       % false: boxed links; true: colored links
    linkcolor=red,          % color of internal links (change box color with 
%linkbordercolor)
    citecolor=green,        % color of links to bibliography
    filecolor=magenta,      % color of file links
    urlcolor=cyan           % color of external links
}


%% Comments
\newif\ifcomments\commentstrue

\ifcomments
\newcommand{\authornote}[3]{\textcolor{#1}{[#3 ---#2]}}
\newcommand{\todo}[1]{\textcolor{red}{[TODO: #1]}}
\else
\newcommand{\authornote}[3]{}
\newcommand{\todo}[1]{}
\fi

\newcommand{\wss}[1]{\authornote{magenta}{SS}{#1}}
\newcommand{\nd}[1]{\authornote{blue}{ND}{#1}}
\newcommand{\ah}[1]{\authornote{violet}{AH}{#1}}

\newcommand{\NN}[1]{{\color{red}#1}}
\newcommand{\WSS}[1]{{\color{blue}#1}}

\newcommand{\colZwidth}{1.0\textwidth}
\newcommand{\blt}{- } %used for bullets in a list
\newcommand{\colAwidth}{0.13\textwidth}
\newcommand{\colBwidth}{0.82\textwidth}
\newcommand{\colCwidth}{0.1\textwidth}
\newcommand{\colDwidth}{0.05\textwidth}
\newcommand{\colEwidth}{0.8\textwidth}
\newcommand{\colFwidth}{0.17\textwidth}
\newcommand{\colGwidth}{0.5\textwidth}
\newcommand{\colHwidth}{0.28\textwidth}
\newcounter{defnum} %Definition Number
\newcommand{\dthedefnum}{GD\thedefnum}
\newcommand{\dref}[1]{GD\ref{#1}}
\newcounter{datadefnum} %Datadefinition Number
\newcommand{\ddthedatadefnum}{DD\thedatadefnum}
\newcommand{\ddref}[1]{DD\ref{#1}}
\newcounter{theorynum} %Theory Number
\newcommand{\tthetheorynum}{T\thetheorynum}
\newcommand{\tref}[1]{T\ref{#1}}
\newcounter{tablenum} %Table Number
\newcommand{\tbthetablenum}{T\thetablenum}
\newcommand{\tbref}[1]{TB\ref{#1}}
\newcounter{assumpnum} %Assumption Number
\newcommand{\atheassumpnum}{P\theassumpnum}
\newcommand{\aref}[1]{A\ref{#1}}
\newcounter{goalnum} %Goal Number
\newcommand{\gthegoalnum}{P\thegoalnum}
\newcommand{\gsref}[1]{GS\ref{#1}}
\newcounter{instnum} %Instance Number
\newcommand{\itheinstnum}{IM\theinstnum}
\newcommand{\iref}[1]{IM\ref{#1}}
\newcounter{reqnum} %Requirement Number
\newcommand{\rthereqnum}{P\thereqnum}
\newcommand{\rref}[1]{R\ref{#1}}
\newcounter{lafnum}
\newcommand{\lthelafnum}{L\thelafnum}
\newcommand{\lref}[1]{L\laf{#1}}
\newcounter{uahnum}
\newcommand{\utheuahnum}{U\theuahnum}
\newcommand{\uref}[1]{U\uaf{#1}}
\newcounter{perfnum}
\newcommand{\ptheperfnum}{P\theperfnum}
\newcommand{\perf}[1]{P\perf{#1}}
\newcounter{oaenum}
\newcommand{\otheoaenum}{O\theoaenum}
\newcommand{\oae}[1]{P\oae{#1}}
\newcounter{masnum}
\newcommand{\mthemasnum}{M\themasnum}
\newcommand{\mas}[1]{P\mas{#1}}
\newcounter{secunum}
\newcommand{\sthesecunum}{S\thesecunum}
\newcommand{\secu}[1]{S\secu{#1}}
\newcounter{culnum}
\newcommand{\ctheculnum}{C\theculnum}
\newcommand{\cul}[1]{P\cul{#1}}
\newcounter{legalnum}
\newcommand{\lthelegalnum}{L\thelegalnum}
\newcommand{\legal}[1]{L\legal{#1}}

\newcounter{ucnum} %Use Case number
\newcommand{\utheucnum}{UC\theucnum}
\newcommand{\ucref}[1]{UC\ref{#1}}


\newcounter{lcnum} %Likely change number
\newcommand{\lthelcnum}{LC\thelcnum}
\newcommand{\lcref}[1]{LC\ref{#1}}

\newcommand{\tclad}{T_\text{CL}}
\newcommand{\degree}{\ensuremath{^\circ}}
\newcommand{\progname}{PolyHarmonics}
\newcommand{\euler}{e}
\newcommand{\complex}{i}

%\oddsidemargin 0mm
%\evensidemargin 0mm
%\textwidth 160mm
%\textheight 200mm
\usepackage{fullpage}

\begin{document}

\title{Software Requirements Specification for \progname} 
\author{Nolan Driessen}
\date{\today}
	
\maketitle

\tableofcontents

\section{Reference Material}

This section records information for easy reference.

\subsection{Table of Units}

Throughout this document SI (Syst\`{e}me International d'Unit\'{e}s) is employed
as the unit system.  In addition to the basic units, several derived units are
used as described below.  For each unit, the symbol is given followed by a
description of the unit with the SI name in parentheses.
\newline 

\renewcommand{\arraystretch}{1.2}
%\begin{table}[ht]
  \noindent \begin{tabular}{l l l} 
    \toprule		
    \textbf{symbol} & \textbf{unit} & \textbf{SI}\\
    \midrule 
    \si{\hertz} & frequency & hertz\\
    \si{\radian} &	angle	& radians\\
    \si{\second} & time & second\\
	\si{\volt} & voltage & volt\\
    \bottomrule
  \end{tabular}
  %	\caption{Provide a caption}
%\end{table}

\subsection{Table of Symbols}

The table that follows summarizes the symbols used in this document along with 
their units.  The symbols are listed in alphabetical order.

\renewcommand{\arraystretch}{1.2}
%\noindent \begin{tabularx}{1.0\textwidth}{l l X}
\noindent \begin{longtable*}{l l p{12cm}} \toprule
  \textbf{symbol} & \textbf{unit} & \textbf{description}\\
  \midrule 
    $A$		& \si{\volt}					  & Amplitude\\
  $\euler$	& \si[per-mode=symbol] {unitless} & Euler's number \\
	$f$		&	Hz							  & Frequency\\
  $\complex$  & \si[per-mode=symbol] {unitless} & Imaginary unit, $\complex = 
\sqrt{-1}$\\
  $k$		&	unitless						& Index of input signal\\
  $n$		&	unitless						& Index of summation\\
  $N$			& unitless						  & Number of discrete samples taken\\
  $t$		&	\si{\second}					& Time\\
  $X_k$		& unitless						  & Complex number encoding phase and amplitude of 
$x_n$\\
  $x_n$		& unitless						  & Time wave input for the Discrete Fourier 
Transform 
(DFT)
  \\
  \bottomrule
\end{longtable*}
\ah{Are the units for Xk and Xn volts?}

\nd{I thought so but after asking around and reading some things im unsure. What are the units of a wave? For $x_n$ the Y axis is voltage the X axis is time, but what would the units of the wave be?}
\subsection{Abbreviations and Acronyms}

\renewcommand{\arraystretch}{1.2}
\begin{tabular}{l l}
  \toprule		
  \textbf{symbol} & \textbf{description}\\
  \midrule 
  A & Assumption\\
  DD & Data Definition\\
  DFT & Discrete Fourier Transform\\
  GS & Goal Statement\\
  LC & Likely Change\\
  R & Requirement\\
  SRS & Software Requirements Specification\\
  T & Theoretical Model\\
%  TDMS & Labview Binary Measurement File\\
  \bottomrule
\end{tabular}\\

\section{Introduction}

A system is needed to efficiently and correctly interpret a signal to relate its 
properties to damage in plastic parts.
%TODO	more detail here
The following section provides an overview of the Software Requirements
Specification (SRS) for \progname.
This section explains the purpose the document is designed to fulfill, the scope 
of the requirements and the organization of the document: what the document is 
based on and intended to portray.

\subsection{Purpose of Document}

The main purpose of this document is to describe the analysis of acoustic 
signals.
The goals and theoretical models used in the 
\progname{} code are provided, with an emphasis on explicitly identifying 
assumptions and unambiguous definitions.  This document is intended to be used 
as a reference to provide all information necessary to 
understand and verify the analysis.  The SRS is abstract because the contents 
say
\emph{what} problem is being solved, but not \emph{how} to solve it.

This document will be used as a starting point for subsequent development
phases, including writing the design specification and the software verification
and validation plan.  The design document will show how the requirements are to
be realized, including decisions on the numerical algorithms and programming
environment.  The verification and validation plan will show the steps that will
be used to increase confidence in the software documentation and the
implementation.  

\subsection{Scope of Requirements} 

The scope of the requirements includes analyzing an input signal and
transforming it to an understandable form. Given a series of LabVIEW files, 
\progname{} is intended to interpret the data within each and output  
waveforms as well as text files that can be related to damage in plastic parts.

\subsection{Organization of Document}

The organization of this document follows the template for an SRS for scientific
computing software proposed by~\cite{Koothoor2013} and \cite{SmithAndLai2005},
with some aspects taken from Volere template 16 \cite{Volere}. The presentation 
follows the
standard pattern of presenting goals, theories, definitions, and assumptions.
For readers that would like a more bottom up approach, they can start reading
the data definitions in Section \ref{sec_datadef} and trace back to find any
additional information they require.  The data definitions provide the
definition of the DFT and algebraic equations that describe the analysis of
signals.  \progname{} solves these equations.

The goal statements are refined to the theoretical models, and theoretical
models to the data definitions.

%\subsection{Intended Audience}

\section{Stakeholders}

This section describes the stakeholders: the people who have an interest in the 
product.

\subsection{The Client}

The client for \progname{} is Dr.~Michael Thompson. The client has the final say
on acceptance of the product.

\subsection{The Customer}

The customers are the end users of \progname{}. They will be senior
undergraduate and graduate level students of McMaster University, studying in
Chemical Engineering.

\section{General System Description}

This section provides general information about the system,
identifies the interfaces between the system and its environment, and describes 
the
user characteristics and the system constraints.

%\subsection{System Context}

\subsection{User Characteristics}
\begin{itemize}
\item The end user of \progname{} is expected to be a senior undergraduate or
 graduate level student in Chemical Engineering or equivalent.
 \item The end user is expected to have an understanding of LabVIEW and the 
ability to operate it.
 \item The end user is expected to have an understanding of ProMV and the 
ability to operate it.
\end{itemize}
\subsection{System Constraints}

\begin{itemize}
\item \progname{} must function with the LabVIEW \cite{TDMS} software used to 
gather input,
  as well as produce text files in a format ProMV \cite{ProMV} can read.  
%TODO potentially change 2.7
\item \progname{} must be developed for and be compatible with Python version
  2.7 and all associated modules.
\end{itemize}

\wss{What did Felipe say about Python 2 versus 3?  If 3 is an option, then maybe
they should just migrate to it now?}

\section{Scope of the Project}

This section presents the scope of the project. It describes the expected use of
\progname{} as well as the inputs and outputs of each action.

\subsection{Product Use Case Table}
\label{UseCase}
\renewcommand{\arraystretch}{1.2}
\captionof{table}{Use Case Table} 
  \noindent \begin{tabular}{l l l l} 
    \toprule		
    \textbf{Use Case No.} & \textbf{Use Case Name} & \textbf{Actor} & 
\textbf{Input and Output}\\
   \midrule
\label{UC_RecordSig}   \refstepcounter{ucnum}\theucnum  & Record Signals 				& 
User 
			& Frequency \newline(input)\\
     &		 				  		&				& Acoustic Signal (output)\\ \midrule
\label{UC_InputFile}   \refstepcounter{ucnum}\theucnum  & Input	  				& User			
& File Directory (input)\\ 
	&								&				& Starting Frequency (input)\\
	&								&				& Stopping Frequency (input)\\
	&								&				& Frequency Step (input)\\
	\midrule

\label{UC_Filter} \refstepcounter{ucnum}\theucnum		& Filter				&
 \progname{}	& TDMS File set (input)\\
 	&								&				& Filtered Data (output)\\ \midrule

\label{UC_AnalyzeInput}   \refstepcounter{ucnum}\theucnum  & Analyze				& 
\progname{}	& TDMS File set (input)\\
   	 &								&				& Filtered Data (input)\\
   	 &								&				& Transformed Signals (output)\\ \midrule
\label{UC_PlotTrans}   \refstepcounter{ucnum}\theucnum  & Output Transformed 
Signal	
& \progname{}	& Transformed Signals (input)\\
   	 &								&				& Plots (output)\\
   	 &								&				& Text files (output)\\ \bottomrule
  \end{tabular}

\subsection{Individual Product Use Cases}

\begin{itemize}
\item \hyperref[UC_RecordSig]{Use Case 1} refers to the collection of data for
  \progname{} to analyze. A specific set of tests is done with a starting
  frequency, ending frequency and step size between each trial. Each of these
  frequencies will be tested for a signal and if a signal is detected it will  
  be written into a TDMS file. If no signal is detected the system will  
  not write to the file, and proceed to the next test.  
\item \hyperref[UC_InputFile]{Use Case 2}
 refers to the user providing input
 to \progname{}. Once the user has input a directory 
\progname{} shall search the directory for TDMS files. The starting, stopping
and step frequencies are also input for use within the analysis.

\item \hyperref[UC_Filter]{Use Case 3} describes the filtering of the original
 signals found with the TDMS Files. The input found through
 \hyperref[UC_InputFile]{Use Case 2} is filtered before any analysis is done.

\item \hyperref[UC_AnalyzeInput]{Use Case 4} describes the analysis being
  done. \progname{} will take each given input and transform the signal it
  contains into the frequency spectrum. This is done for both the original
  signals and the filtered signals.

\item \hyperref[UC_PlotTrans]{Use Case 5} gives output to the user in the form
  of a series of plots and three text files. Each TDMS file creates two plots, 
one amplitude vs time and another amplitude vs frequency. Additionally 
one contour plot is created for the directory.
\end{itemize}

\section{Specific System Description}

This section first presents the problem description, which gives a high-level
view of the problem to be solved.  This is followed by the solution
characteristics specification, which presents the assumptions, theories and
definitions that describe the analysis of the signals.

\subsection{Problem Description} \label{Sec_pd}

A system is needed to efficiently and correctly interpret a signal. \progname{}
is a computer program developed to interpret acoustic data and produce a
waveform that can be related to damage in plastic parts.

%\subsubsection{Background}

\subsubsection{Terminology and  Definitions}

This subsection provides a list of terms that are used in the subsequent
sections and their meaning, with the purpose of reducing ambiguity and making it
easier to correctly understand the requirements:

\begin{itemize}

\item Radian: A unit of angle, equal to an angle at the center of a circle whose
  arc is equal in length to the radius.
\item Acoustics: The science of studying mechanical waves in gases, liquids and
  solids. Specifically \progname{} refers to the study of sound waves in solids.
\item Fit Criterion: A benchmark to objectively determine whether the
  implemented product has properly met the requirements.
\item LabVIEW: A separate software program that creates TDMS files for 
\progname{} to analyze.
\item ProMV: A separate software program that receives the text files
  \progname{} produces as its input. Performs further analysis on the signals.
\item TDMS: Labview Binary Measurement File. More information can be found here 
\cite{TDMS}\\
\item System as is: Implementation of PolyHarmonics created before June 3rd 
2015.
\item System to be: The system that will be created through the use of this 
documentation.
\item Recorded Frequency: The frequency LabVIEW received and stored.
\item Emitted Frequency: The frequency sent into the plastic part
\end{itemize}

\subsubsection{Physical System Description}

The physical system involves testing a plastic part for damage. A signal is sent 
through a plastic part by an emitter and taken in by a receiver. The frequency 
of the received signal is then analyzed.

\subsubsection{Goal Statements}

\begin{itemize}
\item[GS\refstepcounter{goalnum}\thegoalnum:] Analyze an acoustic signal by 
receiving a directory containing one or more TDMS files as input and returning
details of the transformed signal that can be related to damage in the
 plastic part.
\end{itemize}

\subsection{Solution Characteristics Specification}

\subsubsection{Assumptions} \label{Assumptions}

This section simplifies the original problem and helps in developing the
theoretical model by filling in the missing information for the physical
system. The numbers given in the square brackets refer to the data definition,
or the instance model, in which the respective assumption is used.

\begin{itemize}

\item[A\refstepcounter{assumpnum}\theassumpnum \label{A_MinVolt}:] Any signal
  with an amplitude below \hyperref[AppendA]{MIN\_AMP} is noise. 
\item[A\refstepcounter{assumpnum}\theassumpnum
\label{A_MaxFreq}:]Any frequencies above \hyperref[AppendA]{MAX\_FREQ} are not
relevant and will not be considered for analysis.

\end{itemize}

\subsubsection{Theoretical Models}\label{sec_theoretical}

This section focuses on the general equations and laws that \progname{} is based
on.
~\newline

\noindent
\begin{minipage}{\textwidth}
\renewcommand*{\arraystretch}{1.5}
\begin{tabular}{| p{\colAwidth} | p{\colBwidth}|}
  \hline
  \rowcolor[gray]{0.9}
  
Number& T\refstepcounter{theorynum}\thetheorynum   \label{T1} \\  
  \hline

  Label&\bf Discrete Fourier Transform\\
  \hline
  Input & The signal to be transformed, $x_n$ \\
  \hline
  Output & $X_k$\\
  \hline
  Equation&  $X_k = \sum\limits_{n=0}^{N-1} x_n e^\frac{-2 \pi \complex k 
            n}{N}$\\   

  \hline
  Description & 
                The above equation defines
                the Discrete Fourier Transform. The DFT converts discrete data 
from a time 
                wave into the frequency spectrum. The meaning of $e^{ix}$ 
is provided in~DD\ref{DD1}.

  $k$ = Index of input signal
  
$n$ = Index of summation

  $N$ = Number of discrete samples taken

  $x_n$ = A wave of Amplitude vs time to be converted to the frequency spectrum
  
  $X_k$ = A complex number encoding both phase and amplitude of $x_n$
  \\
  \hline
  Source &
  \href{http://en.wikipedia.org/wiki/Discrete\_Fourier\_transform}{  
http://en.wikipedia.org/wiki/Discrete\_Fourier\_transform}\\

  \hline
  Ref. By\ & \\
  \hline
\end{tabular}
\end{minipage}\\

\noindent
\begin{minipage}{\textwidth}
\renewcommand*{\arraystretch}{1.5}
\begin{tabular}{| p{\colAwidth} | p{\colBwidth}|}
  \hline
  \rowcolor[gray]{0.9}
  
Number& T\refstepcounter{theorynum}\thetheorynum   \label{T2} \\  
  \hline

  Label&\bf Haar Wavelet Filter\\
  \hline
  Input & The signal to be filtered \\ %TODO	put a variable representation here
  \hline
  Output & \\
  \hline
  Equation&  \\   

  \hline
  Description & 
                
  \\
  \hline
  Source &
  \\

  \hline
  Ref. By\ & \nd{Having trouble finding any information on this}
\wss{Can you ask Felipe or Trevor?}\\
  \hline
\end{tabular}
\end{minipage}\\

\subsubsection{Data Definitions}\label{sec_datadef}

This section collects and defines all the data needed to fully define the 
system.

~\newline

\noindent
\begin{minipage}{\textwidth}
\renewcommand*{\arraystretch}{1.5}
\begin{tabular}{| p{\colAwidth} | p{\colBwidth}|}
\hline
\rowcolor[gray]{0.9}

Number& DD\refstepcounter{datadefnum}\thedatadefnum \label{DD1} \\
\hline 

Label& \bf Euler's Formula\\
\hline
Equation&$e^{\complex x} = \cos{x} + \complex \sin{x}$\\
\hline
Description & 
Euler's formula establishes the relationship between 
trigonometric functions and the complex exponential function.

$x$ = Radian input
\\
\hline
Sources& \\
\hline
Ref.\ By & T\ref{T1}\\
\hline
\end{tabular}
\end{minipage}\\

~\newline

~\newline
\subsubsection{Data Constraints} \label{sec_DataConstraints}    

Table~\ref{TblInputVar} and \ref{TblOutputVar} show the data constraints on the
input and output variables, respectively.  The column physical constraints gives
the physical limitations on the range of values that can be taken by the
variable.  The constraints are conservative, to give the user of the model the
flexibility to experiment with unusual situations.  The column of typical values
is intended to provide a feel for a common scenario.  The uncertainty column
provides an estimate of the confidence with which the physical quantities can be
measured.  This information would be part of the input if one were performing an
uncertainty quantification exercise.


\begin{table}[!h]
\caption{Input Variables} \label{TblInputVar}
\renewcommand{\arraystretch}{1.2}
\noindent \begin{longtable*}{l l l} 
  \toprule
  \textbf{Var} & \textbf{Physical Constraints} & \textbf{Typical Value}\\
  \midrule 
  $N$ & $N>0$ \\
  Starting Frequency & $f \leq$ \hyperref[AppendA]{MAX\_FREQ} & 100kHz\\
  Stopping Frequency  & $f \leq$ \hyperref[AppendA]{MAX\_FREQ} & 1000kHz\\
  Step Frequency & $f \leq$ \hyperref[AppendA]{MAX\_FREQ} & 100kHz\\
  $x$ & $0 \leq x \leq 2 \pi$ & \\
  $x_n$ & $f \leq$ \hyperref[AppendA]{MAX\_FREQ} (A\hyperref[A_MaxFreq]{2}), $A 
\geq$ 
\hyperref[AppendA]{MIN\_AMP} (A\hyperref[A_MinVolt]{1}) &\\  
  \bottomrule
\end{longtable*}
\end{table}

\noindent \begin{description}
\item[(*)] These quantities cannot be equal to zero, or there will be a divide
  by zero in the model.
\end{description}

\begin{table}[!h]
\caption{Output Variables} \label{TblOutputVar}
\renewcommand{\arraystretch}{1.2}
\noindent \begin{longtable*}{l l} 
  \toprule
  \textbf{Var} & \textbf{Physical Constraints} \\
  \midrule 
  $A$ & None\\
  $f$ & None\\
  \bottomrule
\end{longtable*}
\end{table}

\section{Functional Requirements} \label{Func}

The following section provides the functional requirements, the business tasks
that the software is expected to complete.  

%\subsection{Functional Requirements}	% You do not want a section to have only
%one subsection.  Moreover, you don't want to repeat the section title in a
%subsection - SS

\noindent \begin{itemize}

\item[R\refstepcounter{reqnum}\thereqnum \label{R_Input}:] 

\textbf{Description:} The system shall accept a directory containing a set of 
LabVIEW TDMS files as
input, each containing a frequency. The user shall input the starting
 frequency of the set, the stopping frequency and frequency step between each 
 input. This information will be used in the creation of the plots and 
 text files.\\
\textbf{Rationale:} The system requires signals to analyze which are stored
within files recorded by LabVIEW.\\
\wss{Did you determine what the system is doing with the user input starting
  frequency, stopping frequency etc?}\\
\ah{Same question. I am still confused how the start/stop/step are related with
 the program? Do different .TDMS files get picked depending on the start
 /stop/step?}\\
\nd{Where the information is used was added, does it sound suitable for this
 section? Roughly speaking it is used for the \emph{name} of the plots + 
 used within ProMV}
%Fit Criterion: \progname{} shall find a number of TDMS files equal to the 
%number of TDMS files stored within all subdirectories of the specified 
%directory.

\item[R\refstepcounter{reqnum}\thereqnum \label{R_Verify}:] 

\textbf{Description:} The product shall verify the input contains a signal
 with an amplitude greater than \hyperref[AppendA]{MIN\_AMP} and a frequency
  below \hyperref[AppendA]{MAX\_FREQ}. Unless both conditions are met 
  the user shall be informed through a warning message. Otherwise the system will 
proceed to the analysis.

\textbf{Rationale:} To avoid any fatal exceptions the contents of all TDMS 
files must be verified.

%Fit Criterion: Each found TDMS file will return a message if the contents do
%not contain proper signal information. If the contents are verified the program
%will continue unhindered.

\item[R\refstepcounter{reqnum}\thereqnum \label{R_Filter}:] 
\textbf{Description:} \progname{} shall use the input files to apply a Haar 
wavelet filter using T\ref{T2}.\\ 
\textbf{Rationale:} 

\item[R\refstepcounter{reqnum}\thereqnum \label{R_Tranform}:] 
\textbf{Description:} \progname{} shall use the input files to calculate the DFT
using T\ref{T1}.\\
\textbf{Rationale:} The signal must be transformed into the frequency spectrum
to properly analyze and interpret the signal.\newline

\wss{Avoid the use of in order to.  You might even want to \textbf{search for 
this
  phrase in your documents}.} \nd{Removed the phrase, bolded your text to ensure 
I remember to search for it}
%Fit Criterion: Each signal found shall have a corresponding frequency spectrum
%signal from which analysis can be done.

\item[R\refstepcounter{reqnum}\thereqnum \label{R_Plot1}:]
\textbf{Description:} Create a plot for each input frequency of the original
 data: $A$ vs $t$.\\
\textbf{Rationale:} The original signal is useful for subsequent analysis.

\item[R\refstepcounter{reqnum}\thereqnum \label{R_Plot2}:]
\textbf{Description:} Create a plot for each input frequency of the 
transformed data: $A$ vs $f$.\\
\textbf{Rationale:} With the transformed data, \progname{} can analyze the
 new signal and return the useful information.

\item[R\refstepcounter{reqnum}\thereqnum \label{R_ContourPlot}:]
\textbf{Description:} Create a contour plot for the analyzed directory
 showing Recorded Frequency vs Emitted Frequency with the value of the
amplitude indicated by the colour of the graph.\\
\textbf{Rationale:} The results of each recorded requency are compiled
into one graph for quick interpretation.

\ah{I am confused by this requirement. specifically:  ''represented by the 
colour(canadian spelling btw) of the graph at the intersection point.'''}

\nd{Removed ``at the intersection point". Good catch with color vs colour 
I will have to be more careful with the spell check. If this is still unclear
we will have to discuss it, im having trouble rewording. If it still doesn't
make sense please look at the contour plot within the analysis folder in 
my section of the repository, that should clear anything up so perhaps
you would have idea on how to reword.}
\item[R\refstepcounter{reqnum}\thereqnum \label{R_TextOriginal}:]
\textbf{Description:} Create a text file for the analyzed directory, of 
the original data. This text file contains the emitted frequency, the received
 frequency and the amplitude of the original signal after 
the DFT (T\ref{T1}) is taken. \\
\textbf{Rationale:} The results of the test must be documented in order
to proceed with further analysis through ProMV.

\item[R\refstepcounter{reqnum}\thereqnum \label{R_TextFiltered}:]
\textbf{Description:} Create a text file for the analyzed directory, of the 
filtered data. This text file contains the emitted frequency, received 
frequency and the amplitude of a signal filtered by a Haar wavelet 
which has then had T\ref{T1} applied to it. \\
\textbf{Rationale:} The results of the test must be documented in order
to proceed with further analysis through ProMV.

\item[R\refstepcounter{reqnum}\thereqnum \label{R_TextScattered}:]
\textbf{Description:} Create a text file for the analyzed directory, of the
scattering data. This text file contains the emitted frequency and a ratio
 between the amplitude of a range near the emitted frequency and the amplitude 
 of the transformed original signal. This ratio is used to confirm how well the
received signal kept the emitted frequency. \\
\textbf{Rationale:} The results of the test must be documented in order
to proceed with further analysis through ProMV.

%Fit Criterion: A number of plots equal to the number of TDMS that were found
%shall be produced with information on both frequency and amplitude.

\end{itemize}

\section{Non-functional Requirements}

This section provides the Non-Functional requirements, the requirements that
specify criteria that can be used to judge the operation of a system, as opposed
to the specific behaviours.

\subsection{Look and Feel Requirements}

\subsubsection{Appearance Requirements}
\noindent \begin{itemize}
\item[LF\refstepcounter{lafnum}\thelafnum\label{NF_laf1}:] \progname{} shall
  appear visually intuitive.
						
  \textbf{Fit Criterion:} At least \hyperref[AppendA]{MIN\_PERCENT\_USERS}
  shall intuitively understand how to use the interface to input a directory on
  their first attempt.
\end{itemize}

\subsubsection{Style Requirements}

N/A

\subsection{Usability and Humanity Requirements}

\subsubsection{Ease of Use Requirements}

\noindent \begin{itemize}
\item[UH\refstepcounter{uahnum}\theuahnum\label{NF_uah1}:] \progname{} shall be
  usable for collecting and analyzing data by any senior
  undergraduate or graduate student studying Chemical Engineering or equivalent
  with no training.

  \textbf{Fit Criterion:} At least \hyperref[AppendA]{MIN\_PERCENT\_USERS}
  shall be able to collect data in accordance with \hyperref[UseCase]{Use Case 
1}, and perform an analysis as described by \hyperref[UseCase]{Use Case 
3}, \hyperref[UseCase]{Use Case 4} and \hyperref[UseCase]{Use Case 5}
 with no training.
\end{itemize}

\subsubsection{Personalization and Internationalization Requirements}

N/A

\subsubsection{Learning Requirements}

\noindent \begin{itemize}
\item[UH\refstepcounter{uahnum}\theuahnum\label{NF_LR}:] The product shall be
  usable by any senior undergraduate or graduate student studying Chemical
  Engineering or equivalent on their first attempt.

  \textbf{Fit Criterion:} \hyperref[AppendA]{MIN\_PERCENT\_USERS} shall be able
  to complete an analysis in accordance with all \hyperref[UseCase]{Use Cases}
  on their first attempt.
\end{itemize}

\subsubsection{Understandability and Politeness Requirements}

\noindent \begin{itemize}
\item[UH\refstepcounter{uahnum}\theuahnum\label{NF_UaPR}:] The product shall use
  symbols and words that are naturally understandable by the user community.

  \textbf{Fit Criterion:} 100\% of suitable users shall understand all symbols
  and terminology used within \progname{}.
\end{itemize}

\subsubsection{Accessibility Requirements}

N/A

\subsection{Performance}

\subsubsection{Speed and Latency Requirements}

\noindent \begin{itemize}
\item[PR\refstepcounter{perfnum}\theperfnum\label{NF_SL1}:] Each analysis
 shall be sufficiently fast to not interrupt work flow.

  \textbf{Fit Criterion:} At least \hyperref[AppendA]{MIN\_PERCENT\_USERS} 
  shall be satisfied with the analysis speed of \progname{} when running the
  standard benchmark test cases.
  
\item[PR\refstepcounter{perfnum}\theperfnum\label{NF_SL2}:]Any interaction
  between the user and the product shall respond fast enough to not interrupt 
  work flow.

  \textbf{Fit Criterion:} At least \hyperref[AppendA]{MIN\_PERCENT\_USERS} 
  shall be satisfied with the response speed of \progname{} when running the
  standard benchmark test cases.
\end{itemize}

\subsubsection{Safety-Critical Requirements}

N/A

\subsubsection{Precision or Accuracy Requirements}

\noindent \begin{itemize}
\item[PR\refstepcounter{perfnum}\theperfnum\label{NF_PA}:] \progname{} shall
  have sufficiently small relative error.

  \textbf{Fit Criterion:} The implementation of \progname{} shall have at most
  relative error equal to the system as is when compared to the standard
  benchmark test cases.
\end{itemize}

\subsubsection{Reliability and Availability Requirements}

N/A

\subsubsection{Robustness or Fault-Tolerance Requirements}

\noindent \begin{itemize}
\item[PR\refstepcounter{perfnum}\theperfnum\label{NF_FT}:] The product shall
  identify false input and reject it.

  \textbf{Fit Criterion:} Input that does not fit within the
  \hyperref[Assumptions]{Assumptions} shall be rejected and the user shall 
  be warned.
\end{itemize}

\subsubsection{Capacity Requirements}

N/A

\subsubsection{Scalability or Extensibility Requirements}

N/A

\subsubsection{Longevity Requirements}

\noindent \begin{itemize} 
\item[PR\refstepcounter{perfnum}\theperfnum\label{NF_LG}:] The product shall
  function for at least \hyperref[AppendA]{MIN\_LIFESPAN}.

\textbf{Fit Criterion:} \progname{} shall fulfill all 
\hyperref[Func]{Requirements} for at least\\ \hyperref[AppendA]{MIN\_LIFESPAN}.
\end{itemize}
\ah{what happened here? - I see a lot of spaces between each word. Maybe it was just 
the way it was compiled, if so then ignore this}

\nd{Forced the last word onto a new line, is this bad practice or acceptable 
when the compiling looks poor}
\subsection{Operational and Environmental Requirements}

\subsubsection{Expected Physical Environment}

\noindent \begin{itemize}
\item[OE\refstepcounter{oaenum}\theoaenum\label{NF_PE}:] The product is expected
  to operate within a lab environment.

  \textbf{Fit Criterion:} \progname{} shall fulfill all 
  \hyperref[Func]{Requirements} within a lab environment.

%\item[OE\refstepcounter{oaenum}\theoaenum\label{NF_PE}:] The product shall
%function on any Windows, OSX or Linux based operating systems.

\end{itemize}

\subsubsection{Requirements for Interfacing with Adjacent Systems}

\noindent \begin{itemize}
\item[OE\refstepcounter{oaenum}\theoaenum\label{NF_AS1}:] The product shall
  receive input from a series of files created within LabVIEW.

  \textbf{Fit Criterion:} \progname{} shall receive one or more TDMS files as 
input in accordance with \hyperref[UseCase]{Use Case 2}, if none are found
 the user shall be warned.

\item[OE\refstepcounter{oaenum}\theoaenum\label{NF_AS2}:] The product shall
  produce text files for ProMV to receive as input.

  \textbf{Fit Criterion:} The text files produced shall be of the same format as 
the system as is.
\end{itemize}


\subsubsection{Productization Requirements}

\noindent \begin{itemize}
\item[OE\refstepcounter{oaenum}\theoaenum\label{NF_PR1}:] \progname{} will
  require the installation of multiple Python modules.

  \textbf{Fit Criterion:} \progname{} shall be able to complete all
\hyperref[UseCase]{Use Cases} with the use of these additional modules.


\item[OE\refstepcounter{oaenum}\theoaenum\label{NF_PR2}:] \progname{} shall be
  able to be installed by an untrained user accompanied by nothing other than
  the products user manual.

  \textbf{Fit Criterion:} \hyperref[AppendA]{MIN\_PERCENT\_USERS} shall be able
  to get \progname{} fully functioning within one hour of beginning setup.
\end{itemize}

\subsubsection{Release Requirements}

\noindent \begin{itemize}

%\item[OE\refstepcounter{oaenum}\theoaenum\label{NF_RP1}:] Maintenance releases
%shall be offered to users when a major bug has been discovered.

\item[OE\refstepcounter{oaenum}\theoaenum\label{NF_RP2}:] Each release will not
  cause previous features to fail.

  \textbf{Fit Criterion:} After each release \progname{} shall continue to
   fulfill all \newline \hyperref[Func]{Requirements}.

\end{itemize}

\subsection{Maintainability and Support Requirements}

\subsubsection{Maintenance Requirements}

\noindent \begin{itemize}
\item[MS\refstepcounter{masnum}\themasnum\label{NF_mas1}:] \progname{} must be
  maintainable by people other than the original developers.

  \textbf{Fit Criterion:} The end users of the product must be able to improve,
  fix or add new functionality to \progname{}.

\end{itemize}

\subsubsection{Supportability Requirements}

\noindent \begin{itemize}
\item[MS\refstepcounter{masnum}\themasnum\label{NF_SR1}:] The product shall have
  a user manual accompanying its release.

  \textbf{Fit Criterion:} Upon release of \progname{} a user manual shall also
  be released containing details on installation and setup.
\end{itemize}

\subsubsection{Adaptability Requirements}

\noindent \begin{itemize}
\item[MS\refstepcounter{masnum}\themasnum\label{NF_AR1}:] The product is
  expected to run under Windows, OSX and Linux based operating systems.
%TODO replace this with a specific number from the V&V when it gets there

  \textbf{Fit Criterion:} All \hyperref[UseCase]{Use Cases} will successfully
   run for all platforms using a standard benchmark test case.
\end{itemize}

\subsection{Security Requirements}

\subsubsection{Access Requirements}

N/A

%\noindent \begin{itemize}
%\item[SR\refstepcounter{secunum}\thesecunum\label{NF_sec1}:] 
%\end{itemize}

\subsubsection{Integrity Requirements}

\noindent \begin{itemize}
\item[SR\refstepcounter{secunum}\thesecunum\label{NF_sec2}:] No information
  shall be stored unnecessarily or distributed under any circumstances.

  \textbf{Fit Criterion:} \progname{} shall allocate no space for storage of any
  personal information.
\end{itemize}

\subsubsection{Privacy Requirements}

N/A

\subsubsection{Audit Requirements}

N/A

\subsubsection{Immunity Requirements}

N/A

\subsection{Cultural Requirements}

\noindent \begin{itemize}
\item[CP\refstepcounter{culnum}\theculnum\label{NF_cul1}:] The product shall use
  Canadian spelling where applicable.

\textbf{Fit Criterion:} All non-project specific words within \progname{}
 shall pass a Canadian spell check program.

\item[CP\refstepcounter{culnum}\theculnum\label{NF_cul2}:] The product shall not
  be offensive to religious or ethnic groups

  \textbf{Fit Criterion:} 0\% of users shall feel antagonized from use of the
  product.
\end{itemize}

\subsection{Legal Requirements}

\subsubsection{Compliance Requirements}

N/A

\subsubsection{Standards Requirements}

\noindent \begin{itemize}
\item[LR\refstepcounter{legalnum}\thelegalnum\label{NF_L1}:] The product shall
  comply with all of McMaster University's rules and regulations regarding
  research development.

  \textbf{Fit Criterion:} Each principle outlined here \cite{ResearchPrinciples} 
will be
  followed.
\end{itemize}

\section{Likely Changes}    

\noindent \begin{itemize}

\item[LC\refstepcounter{lcnum}\thelcnum\label{LC_comp}:] Currently only
  frequency is being analyzed, however it is likely that in the future different
  components of the system, such as the number and time between pulses, will
  also be taken into account.
\item[LC\refstepcounter{lcnum}\thelcnum\label{LC_signal}:] As new information
  becomes relevant different methods might be used to analyze and interpret the
  signal.
\item[LC\refstepcounter{lcnum}\thelcnum\label{LC_plot}:] The number and 
contents of the plots that are created from the transformed signal are likely 
to be represented in different ways in the future. In the event this changes,
the \hyperref[UseCase]{UseCases} and \hyperref[Func]{Functional Requirements}
will require updating.
\item[LC\refstepcounter{lcnum}\thelcnum\label{LC_output}:]The number and 
contents of the produced text files. In the event of a change,
the \hyperref[UseCase]{UseCases} and \hyperref[Func]{Functional Requirements}
will require updating.
\item[LC\refstepcounter{lcnum}\thelcnum\label{LC_filter}:] The method of 
filtering may change as different information becomes relevant.
\end{itemize}
\newpage


\bibliographystyle{ieeetr}
\bibliography{PolyHarmonics_SRS}
\nd{Bibliography links dont work, unsure how to handle this}
\appendix

\section{Symbolic Constants} \label{AppendA}

\renewcommand{\arraystretch}{1.2}
\begin{tabular}{l l l}
  \toprule		
  \textbf{constant name} & \textbf{description} & \textbf{Referenced By}\\
  \midrule 
  MAX\_FREQ					&	800kHz		& A\ref{A_MaxFreq}, Table \ref{TblInputVar}, 
R\ref{R_Verify}\\
  MAX\_SECONDS				&	3s			& PR\ref{NF_SL1}, PR\ref{NF_SL2}\\
  MIN\_AMP					&	0.06V		& A\ref{A_MinVolt}, Table \ref{TblInputVar}, 
R\ref{R_Verify}\\
  MIN\_PERCENT\_USERS		&	80\%		& LF\ref{NF_laf1}, UH\ref{NF_uah1}, 
UH\ref{NF_LR}, PR\ref{NF_SL1}, PR\ref{NF_SL2}, OE\ref{NF_PR2}\\
  MIN\_LIFESPAN				&	3 years		& PR\ref{NF_LG}\\
  \bottomrule
\end{tabular}\\

\wss{This file is coming along nicely.  We should be able to send it to the
  customers for comment soon.}

\end{document}










