\documentclass[12pt]{article}

\usepackage{bm}
\usepackage{amsmath}
\usepackage{amsfonts}
\usepackage{amssymb}
\usepackage{graphicx}
\usepackage{colortbl}
\usepackage{xr}
\usepackage{hyperref}
\usepackage{longtable}
\usepackage{xfrac}
\usepackage{tabularx}
\usepackage{float}
\usepackage{siunitx}
\usepackage{booktabs}
\usepackage[usenames,dvipsnames]{xcolor}

\hypersetup{
      colorlinks=true,       % false: boxed links; true: colored links
    linkcolor=red,          % color of internal links (change box color with 
%linkbordercolor)
    citecolor=green,        % color of links to bibliography
    filecolor=magenta,      % color of file links
    urlcolor=cyan           % color of external links
}


%% Comments
\newif\ifcomments\commentstrue

\ifcomments
\newcommand{\authornote}[3]{\textcolor{#1}{[#3 ---#2]}}
\newcommand{\todo}[1]{\textcolor{red}{[TODO: #1]}}
\else
\newcommand{\authornote}[3]{}
\newcommand{\todo}[1]{}
\fi

\newcommand{\wss}[1]{\authornote{magenta}{SS}{#1}}
\newcommand{\nd}[1]{\authornote{blue}{ND}{#1}}
\newcommand{\ah}[1]{\authornote{violet}{AH}{#1}}

\newcommand{\progname}{PolyHarmonics}

\usepackage{fullpage}

\begin{document}

\title{User Manual for \progname} 
\author{Nolan Driessen}
\date{\today}
	
\maketitle

\tableofcontents

\section{Introduction}

This document is designed to assist with installation, setup and upkeep of 
\progname{} 
on Python version 2.7.


\section{Installation}
The following section provides an overview of what needs to be installed before 
\progname{} will function.

\subsection{Python}
\progname{} is designed to function on Python version 2.7, as such Python must 
be installed and updated to this version in order for the software to run 
properly.
\subsection{Required Packages}
The following packages do not come with Python 2.7 by default and must be 
installed separately.
\begin{itemize}
\item NumPy: Provides support for large multi-dimensional arrays and many high 
level mathematical functions.
\item SciPy: Builds on NumPy with modules for optimization and many commonly 
used functions in scientific computing.
\item Matplotlib: A Python 2D plotting library.
\item Pywt: Used for wavelet transforms, also known as PyWavelets.
\item npTDMS: Required in order to read TDMS files.
\end{itemize}

\subsection{Installation Steps} \label{Easy}
The following method requires pip which comes preinstalled with Python version 
2.7.9 and beyond. 
\begin{itemize}
\item From the command line type the following command:
pip install [package name]

\item To verify the package was properly installed, open an Idle Python GUI and 
enter the command:
import [package name]
\item If no error is returned the packages have been properly installed. However 
if an import error is given, please refer to section \ref{Detailed} for further 
instructions.
\end{itemize}
\subsubsection{Detailed Windows Installation Steps} \label{Detailed}
In the event the steps outlined in section \ref{Easy} fail follow these steps:
\begin{itemize}
\item From a command line enter the command: pip install update
\item From either a Python shell or command line running Python run the 
following command: import pip; print pip.pep425tags.get\_supported()

Take note of the first entry as this will be used to install the proper 
modules for your system.
\item From the following link download each required package according to the 
first entry returned from the previous step. Please note this website is 
unofficial. \newline
\url{http://www.lfd.uci.edu/~gohlke/pythonlibs/}

For example, if the first entry returned from the python command was (`cp27',
`none', `win32') you would want the NumPy the file called 
``numpy‑1.9.2+mkl‑cp27‑none‑win32.whl". Download the proper version of each
required module and place it within the\\
Python27/Tools/Scripts folder.
\item Navigate a command window to the Python/Tools/scripts folder and run the 
command: pip install your-package.whl  
\newline Where your-package is the name of the 
file previously downloaded.
\end{itemize}


\section{Using \progname{}}

\subsection{Running}
Once the specified modules have been properly installed \progname{} can be run 
by opening the file main.py and choosing run. \\
Before running, the file config.txt must be modified to contain proper input. 
This file is located in the same directory as main.py. Open config.txt and edit 
the start frequency, stop frequency and step frequency as well as the Main 
Directory entries. The system will use these entries as input and give all 
output in the location specified by Main Directory. 

\section{Likely Changes}
\progname{} has been developed in order to make likely changes easily. 
When making a change, first refer to the Module Guide in order to identify which 
module will contain this change. 
\subsubsection{New Functions}
In some cases, a new function will need to be created in order to accommodate a 
change. If this is the case, within the proper module create a new function and 
identify which input will be required from other modules. Using the Get function 
of that specific data you can receive the input from a separate module. Once the 
required input has been gathered the body of the function can be built according 
to the problem being solved. Finally, after the function is complete main.py 
must be modified to call this function, have main.py call the function using dot 
notation, module.function().
\subsubsection{Modifying Existing Functions}
In the case an existing function must be modified the steps will be similar to 
creating a new function. First identify the module that the change should be 
made in. If a function already exists but must be modified, identify if the 
required input is already within the module. If necessary remove unnecessary Get 
functions, or add the new required ones. Finally modify the function 
accordingly, no changes should need to be made to main.py or any other module if 
the likely change is well contained.

\section{Testing}
This section describes running the test cases for \progname{}, including setup 
and expected behaviour. 
\subsection{System Tests}
There is one system test created at the moment. In order to run the test, ensure 
that the config.txt file is setup such that:\\
Start Frequency: 100\\
Stop Frequency: 1000\\
Step Frequency: 100\\
Main Directory: standard\_test\_cases/input 1\\
One additional package is required for the system testing, download Pillow using the steps outlined in section \ref{Detailed}. Once the parameters are set, open Testing.py and run it. The system will run 
main.py and execute one test case, which compares the created text files and 
graphs to the expected output. If the system either errors or fails the program 
is not running as expected and must be fixed before being used for analysis.
\subsection{Unit Tests}
Unit testing tests each module on its own, rather than the whole system at once. 
The unit tests do not require any modification of config.txt. In order to run 
the unit tests, open unitTest.py and run it. This will execute 6 tests, and 
indicate if any error or fail. 

\subsection{Python Version 3}
In order to convert the current Python 2.7 code into Python 3 code open a 
command line, navigate to the folder containing the \progname{} modules and run 
the following command:\\
python C:/Python27/Tools/scripts/2to3.py -w [module to convert]\\
Where [module to convert] will be replaced by one of the files, such as main.py.
This will have to be done for each module, and the result will have a .bak 
backup file and the Python 3 version of the converted code. 

\end{document}







