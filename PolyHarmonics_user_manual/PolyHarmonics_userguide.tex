\documentclass[12pt]{article}

\usepackage{bm}
\usepackage{amsmath}
\usepackage{amsfonts}
\usepackage{amssymb}
\usepackage{graphicx}
\usepackage{colortbl}
\usepackage{xr}
\usepackage{hyperref}
\usepackage{longtable}
\usepackage{xfrac}
\usepackage{tabularx}
\usepackage{float}
\usepackage{siunitx}
\usepackage{booktabs}
\usepackage[usenames,dvipsnames]{xcolor}

\hypersetup{
      colorlinks=true,       % false: boxed links; true: colored links
    linkcolor=red,          % color of internal links (change box color with 
%linkbordercolor)
    citecolor=green,        % color of links to bibliography
    filecolor=magenta,      % color of file links
    urlcolor=cyan           % color of external links
}


%% Comments
\newif\ifcomments\commentstrue

\ifcomments
\newcommand{\authornote}[3]{\textcolor{#1}{[#3 ---#2]}}
\newcommand{\todo}[1]{\textcolor{red}{[TODO: #1]}}
\else
\newcommand{\authornote}[3]{}
\newcommand{\todo}[1]{}
\fi

\newcommand{\wss}[1]{\authornote{magenta}{SS}{#1}}
\newcommand{\nd}[1]{\authornote{blue}{ND}{#1}}
\newcommand{\ah}[1]{\authornote{violet}{AH}{#1}}

\newcommand{\progname}{PolyHarmonics}

\usepackage{fullpage}

\begin{document}

\title{User Manual for \progname} 
\author{Nolan Driessen}
\date{\today}
	
\maketitle

\tableofcontents

\section{Introduction}

\subsection{Purpose of Document}

This document is designed to assist with installation and setup of \progname{} 
on Python version 2.7.


\section{Installation}
The following section provides an overview of what needs to be installed before 
\progname{} will function.

\subsection{Python}
\progname{} is designed to function on Python version 2.7, as such Python must 
be installed and updated to this version in order for the software to run 
properly.
\subsection{Required Modules}
The following modules do not come with Python 2.7 by default and must be 
installed separately.
\begin{itemize}
\item NumPy: Provides support for large multi-dimensional arrays and many high 
level mathematical functions.
\item SciPy: Builds on NumPy with modules for optimization and many commonly 
used functions in scientific computing.
\item Matplotlib: A Python 2D plotting library.
\item Pywt: Used for wavelet transforms, also known as PyWavelets.
\item npTDMS: Required in order to read TDMS files.
\end{itemize}

\subsection{Installation Steps}
In order to install the modules properly follow these steps:
\begin{itemize}
\item From a command line enter the command: pip install update
\item From either a Python shell or command line running Python run the 
following command: import pip; print(pip.pep425tags.get\_supported())

Take note of the first entry as this will be used to install the proper 
modules for your system.
\item From the following link download each required modules according to the 
first entry returned from the previous step. Please note this website is unofficial. \newline
\url{http://www.lfd.uci.edu/~gohlke/pythonlibs/}

For example, if the first entry returned from the python command was (`cp34',
`none', `win32') you would want the NumPy the file called 
``numpy‑1.9.2+mkl‑cp34‑none‑win32.whl". Download the proper version of each
required module and place it within the \newline
Python/Tools/Scripts folder.
\item Navigate a command window to the Python/Tools/scripts folder and run the 
command: pip install your-package.whl  
\newline Where your-package is the name of the 
file previously downloaded.
\end{itemize}


\section{Using \progname{}}

\subsection{Running}
Once the specified modules have been properly installed \progname{} can be run 
by opening the .py file and choosing run. 
\newline When prompted input the starting frequency, the stopping frequency and 
the step between each frequency that was used to test. Then select the directory 
which contains the folder you wish to use as input. All plots and text files the 
system outputs will go into this folder.





\end{document}


